\documentclass[sigconf]{acmart}
\settopmatter{authorsperrow=4}
\usepackage{booktabs} % For formal tables 

\usepackage{multirow}
%\usepackage[table,xcdraw]{xcolor}
\usepackage[ruled, linesnumbered]{algorithm2e}
\usepackage{algpseudocode}

\usepackage{graphicx}
\usepackage{multirow}
\usepackage{type1cm}
\usepackage{amsmath}
\usepackage{booktabs, threeparttable}
\usepackage{color}
\usepackage{enumitem}
\usepackage{url}
\usepackage{listings}
\usepackage{subcaption}
\usepackage{caption}

\usepackage{balance} % For balanced columns on the last page

\newlist{steps}{enumerate}{1}
\setlist[steps, 1]{label = S\arabic*:}


%\captionsetup[table]{skip=4pt}
%\setlength{\textfloatsep}{3pt}
%\setlength{\intextsep}{2pt}

% Copyright
%\setcopyright{none}
%\setcopyright{acmcopyright}
%\setcopyright{acmlicensed}
%\setcopyright{rightsretained}
%\setcopyright{usgov}
%\setcopyright{usgovmixed}
%\setcopyright{cagov}
%\setcopyright{cagovmixed}


%\setlength{\textheight}{10.1in}
%\setlength{\textwidth}{7.8in} \setlength{\topmargin}{-0.8in}
%\setlength{\oddsidemargin}{-.62in}
%\setlength{\evensidemargin}{-.62in}

% This line disables the ACM reference format
\settopmatter{printacmref=false} 
% removes footnote with conference information in first column
\renewcommand\footnotetextcopyrightpermission[1]{} 

\begin{document}%\sloppy
%\SetAlFnt{\small}

\fancyhead{}
\title{Cpp-Taskflow: Fast Parallel Programming with Task Dependency Graphs}



\author{Chun-Xun Lin}
\affiliation{%
  \institution{Dept. of ECE, UIUC}
  \state{IL}
  \country{USA}
}
\email{clin99@illinois.edu}
%
\author{Tsung-Wei Huang}
\affiliation{%
  \institution{Dept. of ECE, UIUC}
  \state{IL}
  \country{USA}
}
\email{twh760812@gmail.com}
%
\author{Guannan Guo}
\affiliation{%
  \institution{Dept. of ECE, UIUC}
  \state{IL}
  \country{USA}
}
\email{guannan4@gmail.com}
%
\author{Martin D. F. Wong}
\affiliation{%
  \institution{Dept. of ECE, UIUC}
  \state{IL}
  \country{USA}
}
\email{mdfwong@illinois.edu}



% The default list of authors is too long for headers.
%\renewcommand{\shortauthors}{B. Trovato et al.}


\begin{abstract}
Nowadays the CPU has multiple cores and it's becoming common for software 
developers to utilize those computing resources to increase the performance of their applications.
However, parallel programming is much more difficult than writing a sequential program, 
especially when complex parallel patterns exist in the application. 
In this paper, we introduce Cpp-Taskflow \cite{cpp-taskflow}, a header-only library that is implemented in
modern C++17 that aims to help quickly develop a parallel application. Cpp-Taskflow lets 
users build dependency task graphs to express their parallel patterns in an intuitive way 
and it is header-only which enables easier drop-in integration into existing applications.

%is more than having multiple threads 
%executing at the same time, 
%Parallel programming is widely used in software development nowadays for 
%pursuing performance. To implement a parallel application, 


\end{abstract}


\maketitle


\section{Introduction}
\label{sec::introduction}
Parallel computing is getting increasingly prevalent and important as a single
chip now can accomodate multiple processing units and is no doubt a pivotal
component in building high-performance software. 
Parallel computing is also very useful to EDA applications as the recent
trend~\cite{routing}~\cite{stok}\cite{Lu2018} shows the EDA tools gain significantly speedup
by scaling to multicores.  Programming a parallel application is a challenging
task because of the indeterministic nature of the concurrent thread execution.
The parallelism might lead to unexpected result if one does not properly
control the execution flow.  
Therefore, compared to writing sequential code, developers need to devote 
more efforts to correctly implement a parallel program.

C++ is the most widely used programming language for developing EDA
applications due to its high performance and it has included essential
facilities to support parallel computing in the standard library~\cite{cpp-thread}, such as the
\textit{thread} object, \textit{mutex}, \textit{lock} and etc.  The standard
library enables the programmers to build the parallel applications from scratch 
but it is not scalable primarily because of the low-level thread managements.
In addition to the standard library, several libraries are invented to expedite the
the development of parallel program such as OpenMP~\cite{openmp} and Intel
Thread Building Blocks (TBB)~\cite{tbb}.  OpenMP provides users a set of
predefined directives to annotate the desired parallelism and the compiler generates
the parallel code based on the annotations. Intel TBB is a template library that supports 
many APIs for different parallel execution patterns. Although both libraries 
solve the low-level thread management problem of standard library, they require
compiler support and library installation which causes the integration
into existing projects quite inconvenient. Furthermore, both libraries 
are not expressive in API to describe parallel applications especially
when tasks dependencies become complex.
% tbb are not expressive in API to describe parallel applications especially
% when tasks dependencies become complex

In this paper, we present Cpp-Taskflow, a header-only library implemented in 
modern C++17. The main idea of Cpp-Taskflow is to let users describe the 
dependency among tasks as a directed graph and the tasks will be executed in
parallel following the given dependency. This model is very intuitive to use
and is sufficient to express most parallel execution patterns. 
The thread execution is automatically undertaken by Cpp-Taskflow and users only need to
focus on maximizing the parallelism without wrestling with the low-level thread
managements. The library is header-only which allows effortless integration
into existing software.  In following sections, we introduce the programming
model of Cpp-Taskflow and the associated APIs for implementing a parallel
program.


\section{Cpp-Taskflow}



\section{ACKNOWLEDGMENT}




%\bibliographystyle{ACM-Reference-Format}
%\bibliography{sample-bibliography}

%\small 
\bibliographystyle{unsrt}
\bibliography{sigproc}  % sigproc.bib is the name of the Bibliography in this case


\end{document}
