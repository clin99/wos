\documentclass[sigconf]{acmart}
\settopmatter{authorsperrow=4}
\usepackage{booktabs} % For formal tables 

\usepackage{multirow}
%\usepackage[table,xcdraw]{xcolor}
\usepackage[ruled, linesnumbered]{algorithm2e}
\usepackage{algpseudocode}

\usepackage{graphicx}
\usepackage{multirow}
\usepackage{type1cm}
\usepackage{amsmath}
\usepackage{booktabs, threeparttable}
\usepackage{color}
\usepackage{enumitem}
\usepackage{url}
\usepackage{listings}
\usepackage{subcaption}
\usepackage{caption}

\usepackage{balance} % For balanced columns on the last page

\newlist{steps}{enumerate}{1}
\setlist[steps, 1]{label = S\arabic*:}


%\captionsetup[table]{skip=4pt}
%\setlength{\textfloatsep}{3pt}
%\setlength{\intextsep}{2pt}

% Copyright
%\setcopyright{none}
%\setcopyright{acmcopyright}
%\setcopyright{acmlicensed}
%\setcopyright{rightsretained}
%\setcopyright{usgov}
%\setcopyright{usgovmixed}
%\setcopyright{cagov}
%\setcopyright{cagovmixed}


%\setlength{\textheight}{10.1in}
%\setlength{\textwidth}{7.8in} \setlength{\topmargin}{-0.8in}
%\setlength{\oddsidemargin}{-.62in}
%\setlength{\evensidemargin}{-.62in}

% This line disables the ACM reference format
\settopmatter{printacmref=false} 
% removes footnote with conference information in first column
\renewcommand\footnotetextcopyrightpermission[1]{} 

\begin{document}%\sloppy
%\SetAlFnt{\small}

\fancyhead{}
\title{Cpp-Taskflow: Fast Parallel Programming with Task Dependency Graphs}



\author{Chun-Xun Lin}
\affiliation{%
  \institution{Dept. of ECE, UIUC}
  \state{IL}
  \country{USA}
}
\email{clin99@illinois.edu}
%
\author{Tsung-Wei Huang}
\affiliation{%
  \institution{Dept. of ECE, UIUC}
  \state{IL}
  \country{USA}
}
\email{twh760812@gmail.com}
%
\author{Guannan Guo}
\affiliation{%
  \institution{Dept. of ECE, UIUC}
  \state{IL}
  \country{USA}
}
\email{guannan4@gmail.com}
%
\author{Martin D. F. Wong}
\affiliation{%
  \institution{Dept. of ECE, UIUC}
  \state{IL}
  \country{USA}
}
\email{mdfwong@illinois.edu}



% The default list of authors is too long for headers.
%\renewcommand{\shortauthors}{B. Trovato et al.}


\begin{abstract}
Nowadays the CPU has multiple cores and it's becoming common for software 
developers to utilize those computing resources to increase the performance of their applications.
However, parallel programming is much more difficult than writing a sequential program, 
especially when complex parallel patterns exist in the application. 
In this paper, we introduce Cpp-Taskflow \cite{cpp-taskflow}, a header-only library that is implemented in
modern C++17 that aims to help quickly develop a parallel application. Cpp-Taskflow lets 
users build dependency task graphs to express their parallel patterns in a intuitive way 
and it is header-only which enables easier drop-in integration into existing applications.

%is more than having multiple threads 
%executing at the same time, 
%Parallel programming is widely used in software development nowadays for 
%pursuing performance. To implement a parallel application, 


\end{abstract}


\maketitle


\section{Introduction}
\label{sec::introduction}
Parallel computing is getting increasingly prevalent and important as a
single chip now can accomodate multiple processing units. 
Parallel computing is also very useful to EDA applications as recent
trend~\cite{stok}\cite{Lu2018} shows the EDA tools gain significantly speedup
by scaling to multicores.  Programming a parallel application is a challenging
task because of the indeterministic nature of the concurrent thread execution.
The parallelism might lead to unexpected result if one does not properly
control the execution flow.  
Therefore, compared to writing sequential code, developers need to devote 
more efforts to correctly implement a parallel program.
%Therefore, developers need to devote substantial
%efforts to write a parallel applications. 



\section{Parallel programming}



\section{ACKNOWLEDGMENT}




%\bibliographystyle{ACM-Reference-Format}
%\bibliography{sample-bibliography}

%\small 
\bibliographystyle{unsrt}
\bibliography{sigproc}  % sigproc.bib is the name of the Bibliography in this case


\end{document}
