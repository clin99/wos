\documentclass[sigconf]{acmart}
\settopmatter{authorsperrow=3}
\usepackage{booktabs} % For formal tables 

\usepackage{multirow}
%\usepackage[table,xcdraw]{xcolor}
\usepackage[ruled, linesnumbered]{algorithm2e}
\usepackage{algpseudocode}

\usepackage{graphicx}
\usepackage{multirow}
\usepackage{type1cm}
\usepackage{amsmath}
\usepackage{booktabs, threeparttable}
\usepackage{color}
\usepackage{enumitem}
\usepackage{url}
\usepackage{listings}
\usepackage{subcaption}
\usepackage{caption}

\usepackage{balance} % For balanced columns on the last page

\newlist{steps}{enumerate}{1}
\setlist[steps, 1]{label = S\arabic*:}


%\captionsetup[table]{skip=4pt}
%\setlength{\textfloatsep}{3pt}
%\setlength{\intextsep}{2pt}

% Copyright
%\setcopyright{none}
%\setcopyright{acmcopyright}
%\setcopyright{acmlicensed}
%\setcopyright{rightsretained}
%\setcopyright{usgov}
%\setcopyright{usgovmixed}
%\setcopyright{cagov}
%\setcopyright{cagovmixed}


%\setlength{\textheight}{10.1in}
%\setlength{\textwidth}{7.8in} \setlength{\topmargin}{-0.8in}
%\setlength{\oddsidemargin}{-.62in}
%\setlength{\evensidemargin}{-.62in}

% This line disables the ACM reference format
\settopmatter{printacmref=false}

\begin{document}%\sloppy
%\SetAlFnt{\small}

\fancyhead{}
\title{cpp-Taskflow: }



\author{Chun-Xun Lin}
\affiliation{%
  \institution{Dept. of ECE, UIUC}
  \state{IL}
  \country{USA}
}
\email{clin99@illinois.edu}
%
\author{Tsung-Wei Huang}
\affiliation{%
  \institution{Dept. of ECE, UIUC}
  \state{IL}
  \country{USA}
}
\email{twh760812@gmail.com}
%
\author{Martin D. F. Wong}
\affiliation{%
  \institution{Dept. of ECE, UIUC}
  \state{IL}
  \country{USA}
}
\email{mdfwong@illinois.edu}



% The default list of authors is too long for headers.
%\renewcommand{\shortauthors}{B. Trovato et al.}


\begin{abstract}
Nowadays the CPU has multiple cores and it's becoming common for software 
developers to utilize those computing resources to increase the performance of their applications.
However, parallel programming is much more difficult than writing a sequential program, 
especially when complex parallel patterns exist in the application. 
In this paper, we introduce Cpp-Taskflow, a header-only library that is implemented in
modern C++17 that can quickly help develop a parallel application. Cpp-Taskflow lets 
users build dependency task graphs to express their parallel patterns in a intuitive way 
and it is header-only which enables easier drop-in integration into applications.

%is more than having multiple threads 
%executing at the same time, 
%Parallel programming is widely used in software development nowadays for 
%pursuing performance. To implement a parallel application, 


\end{abstract}

%
% The code below should be generated by the tool at
% http://dl.acm.org/ccs.cfm
% Please copy and paste the code instead of the example below.
%
%\begin{CCSXML}
%<ccs2012>
% <concept>
%  <concept_id>10010520.10010553.10010562</concept_id>
%  <concept_desc>Computer systems organization~Embedded systems</concept_desc>
%  <concept_significance>500</concept_significance>
% </concept>
% <concept>
%  <concept_id>10010520.10010575.10010755</concept_id>
%  <concept_desc>Computer systems organization~Redundancy</concept_desc>
%  <concept_significance>300</concept_significance>
% </concept>
% <concept>
%  <concept_id>10010520.10010553.10010554</concept_id>
%  <concept_desc>Computer systems organization~Robotics</concept_desc>
%  <concept_significance>100</concept_significance>
% </concept>
% <concept>
%  <concept_id>10003033.10003083.10003095</concept_id>
%  <concept_desc>Networks~Network reliability</concept_desc>
%  <concept_significance>100</concept_significance>
% </concept>
%</ccs2012>
%\end{CCSXML}
%
%\ccsdesc[500]{Computer systems organization~Embedded systems}
%\ccsdesc[300]{Computer systems organization~Redundancy}
%\ccsdesc{Computer systems organization~Robotics}
%\ccsdesc[100]{Networks~Network reliability}
%
%
%\keywords{ACM proceedings, \LaTeX, text tagging}



%\copyrightyear{2018} 
%\acmYear{2018} 
%\setcopyright{acmcopyright}
%\acmConference[GLSVLSI '18]{2018 Great Lakes Symposium on VLSI}{May 23--25, 2018}{Chicago, IL, USA}
%\acmBooktitle{GLSVLSI '18: 2018 Great Lakes Symposium on VLSI, May 23--25, 2018, Chicago, IL, USA}
%\acmPrice{15.00}
%\acmDOI{10.1145/3194554.3194560}
%\acmISBN{978-1-4503-5724-1/18/05}


\maketitle


\section{Introduction}
\label{sec::introduction}


\section{Parallel programming}


\cite{PG1}

\section{ACKNOWLEDGMENT}




%\bibliographystyle{ACM-Reference-Format}
%\bibliography{sample-bibliography}

%\small 
\bibliographystyle{unsrt}
\bibliography{sigproc}  % sigproc.bib is the name of the Bibliography in this case


\end{document}
